\title[Tucumán - Argentina] % (optional, use only with long paper titles)
{Continous laziness }

\subtitle
{``a realistic software-development new approach``} % (optional)

\author[Pablo A. Frias]           % author name, can use multiple authors (\author[author1, author2])
{
 {Pablo. A.~Frias} \textcolor{black!70}{\inst{1}\inst{2}\inst{3}}
\
}
\institute[Hack-IT  ]      % (optional, but mostly needed)
{

\textcolor{black!70}{\inst{1}}
   Hack-IT\\
   www.hack-it.com.ar
   \and
\textcolor{black!70}{\inst{2}}
   Death threats inbox\\
   pablo@hack-it.com.ar
    \and

 
\textcolor{black!70}{\inst{3}}
   Review later this talk on Github \\ 
   https://github.com/pabl0xf          
}


 \date{ \tiny Buro Coworking \\
    HackerSpace Tucuman  - 
    \textcolor{black!70}{\today}}

\subject{Talks}
% This is only inserted into the PDF information catalog. Can be left
% out. 

% If you have a file called "university-logo-filename.xxx", where xxx
% is a graphic format that can be processed by latex or pdflatex,
% resp., then you can add a logo as follows:

% \pgfdeclareimage[height=0.5cm]{university-logo}{university-logo-filename}
% \logo{\pgfuseimage{university-logo}}
